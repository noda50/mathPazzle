%%----------------------------------------------------------------------
\section{素数の組}
\label{s:素数の組}
%% - - - - - - - - - - - - - - - - - - - - - - - - - - - - - - - - - - -

%%--------------------------------------------------
\subsection{問題}
\label{ssec:素数の組:問題}
%% - - - - - - - - - - - - - - - - - - - - - - - - -

$p^q + q^p$ が素数になる素数の組 $\Tuple{p,q}$ を全て求めよ。
ただし、$p<q$ とする。

\clearpage
%%--------------------------------------------------
\subsection{解答}
\label{ssec:素数の組:解答}
%% - - - - - - - - - - - - - - - - - - - - - - - - -

$p,q$ が共に奇数の素数とする。
その場合、$p^q, q^p$ は共に2より大きい奇数となる。
よって、$p^q + q^p$ は2より大きい偶数になり、素数ではない。

よって、$p,q$ はいずれかは偶数の素数、すなわち $2$ となる。
また、$p < q$ から、$p = 2$ となる。

$q = 3$ の場合、$p^q + q^p = 2^3 + 3^2 = 8 + 9 = 17$ となり、素数である。
よって、 $\Tuple{2,3}$ は条件を満たす組となる。

$q > 3$ の場合、
2の倍数、3の倍数は素数ではないので、
$q = 6k \pm 1$ と表される。
よって、
  %%$$$$$$$$$$$$$$$$$$$$$$$$$$$$$$$$$$$$$$$$$$$$$$$$$$$$$$$$$$$$$$$$$$$$$$
  \begin{eqnarray}
    p^q + q^p
      & = &
        2^{6 k \pm 1} + (6 k \pm 1)^2
  \\
      & = &
        2 + 1   \mod 3
  \\
      & = &
        3
  \\
      & = &
        0   \mod 3
  \end{eqnarray}
  %%$$$$$$$$$$$$$$$$$$$$$$$$$$$$$$$$$$$$$$$$$$$$$$$$$$$$$$$$$$$$$$$$$$$$$$
となり、かならず 3 で割り切れる。
すなわち、素数ではない。

よって、$q > 3$ の場合、条件を満たす組 $\Tuple{p,q}$ は存在しない。

よって、条件を満たす素数の組は、$\Tuple{2,3}$ のみである。
\QED




