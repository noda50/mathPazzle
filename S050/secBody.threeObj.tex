%%----------------------------------------------------------------------
\section{三体に関する問題}
\label{s:三体に関する問題}
%% - - - - - - - - - - - - - - - - - - - - - - - - - - - - - - - - - - -

%%--------------------------------------------------
\subsection{宇宙船の加速}
\label{ssec:宇宙船の加速}
%% - - - - - - - - - - - - - - - - - - - - - - - - -
%%-----------------------------
\subsubsection{問題}
\label{sssec:宇宙船の加速:問題}
%% - - - - - - - - - - - - - -

地球から何光年も離れたある星までの宇宙旅行を考える。
(目的の星は、地球に対し相対速度 0 であるとしておく。)

質量 $m$ の宇宙船を加速し、速度 $v$ まで加速してから巡航し、
その後、減速して目的の星に着陸する。
星での調査の後、地球に向けて同様に加速・減速して帰還する。
この加速・減速に対し、超未来の技術により、
ほぼエネルギーロスゼロで加速・減速できるものとする。

このプランで、出発時に宇宙船と同じ質量 $m$ のエネルギー源を積んでいったとして、
光速 $c$ との速度比 $r = v/c$ の最大値を求めよ。

\paragraph*{ヒント}
相対性理論により、光速を $c$ とすると、
静止質量 $m_0$ の物体が速度 $v$ で動いている場合の質量 $m_v$ は以下の式となる。
  %%$$$$$$$$$$$$$$$$$$$$$$$$$$$$$$$$$$$$$$$$$$$$$$$$$$$$$$$$$$$$$$$$$$$$$$
  \begin{eqnarray}
    m_v & = & \frac{m_0}{\sqrt{1-\frac{v^2}{c^2}}}
  \end{eqnarray}
  %%$$$$$$$$$$$$$$$$$$$$$$$$$$$$$$$$$$$$$$$$$$$$$$$$$$$$$$$$$$$$$$$$$$$$$$

\clearpage
%%------------------------------
\subsubsection{解答}
\label{sssec:宇宙船の加速:解答}
%% - - - - - - - - - - - - - - -

エネルギー用物質の質量を $m_E$ としておく。

出発時の総重量 $m_a$ は、
  %%$$$$$$$$$$$$$$$$$$$$$$$$$$$$$$$$$$$$$$$$$$$$$$$$$$$$$$$$$$$$$$$$$$$$$$
  \begin{eqnarray}
    m_a & = & m + m_E
  \end{eqnarray}
  %%$$$$$$$$$$$$$$$$$$$$$$$$$$$$$$$$$$$$$$$$$$$$$$$$$$$$$$$$$$$$$$$$$$$$$$
往路の加速後の速度 $v$ における静止質量を $m_b$ としておく。
(加速によりエネルギーを使い、質量が減る。)
また、その速度での、地球運動系での質量を $m_c$ とする。
この時、エネルギー変換が理想的であるとすると、以下の式が成り立つ。
  %%$$$$$$$$$$$$$$$$$$$$$$$$$$$$$$$$$$$$$$$$$$$$$$$$$$$$$$$$$$$$$$$$$$$$$$
  \begin{eqnarray}
    m_c & = & m_a
  \\
    m_b & = & m_c \sqrt{1 - \frac{v^2}{c^2}} = m_a \sqrt{1 - r^2}
  \end{eqnarray}
  %%$$$$$$$$$$$$$$$$$$$$$$$$$$$$$$$$$$$$$$$$$$$$$$$$$$$$$$$$$$$$$$$$$$$$$$
目的の星に静止した時の静止質量を $m_d$ とし、
速度 $v$ での運動系における質量を $m_e$ とすると、以下の式となる。
  %%$$$$$$$$$$$$$$$$$$$$$$$$$$$$$$$$$$$$$$$$$$$$$$$$$$$$$$$$$$$$$$$$$$$$$$
  \begin{eqnarray}
    m_e & = & m_b
  \\
    m_d & = & m_e \sqrt{1 - \frac{v^2}{c^2}} = m_b \sqrt{1 - r^2}
  \\
        & = & m_a \sqrt{1 - r^2} \sqrt{1 - r^2} = m_a (1 - r^2)
  \end{eqnarray}
  %%$$$$$$$$$$$$$$$$$$$$$$$$$$$$$$$$$$$$$$$$$$$$$$$$$$$$$$$$$$$$$$$$$$$$$$
これを、復路でも繰り返すので、
復路の速度 $v$ での移動時の静止質量を $v_f$、
地球運動系での質量を $v_g$、
地球に静止した時の静止質量を $v_h$、
速度 $v$ の運動系から見た質量を $v_i$ とすると、
  %%$$$$$$$$$$$$$$$$$$$$$$$$$$$$$$$$$$$$$$$$$$$$$$$$$$$$$$$$$$$$$$$$$$$$$$
  \begin{eqnarray}
    m_g & = & m_d
  \\
    m_f & = & m_g \sqrt{1 - \frac{v^2}{c^2}} = m_d \sqrt{1 - r^2}
  \\
        & = & m_a (1 - r^2) \sqrt{1 - r^2}
  \\
    m_i & = & m_f
  \\
    m_h & = & m_i \sqrt{1 - \frac{v^2}{c^2}} = m_f \sqrt{1 - r^2}
  \\
        & = & m_a (1 - r^2) \sqrt{1 - r^2} \sqrt{1 - r^2} = m_a(1 - r^2)^2
  \end{eqnarray}
  %%$$$$$$$$$$$$$$$$$$$$$$$$$$$$$$$$$$$$$$$$$$$$$$$$$$$$$$$$$$$$$$$$$$$$$$
地球帰還時に、エネルギー用物質を使い果たしているとすると、$m_h = m$ である。
また、仮定により、$m_E = m$ であるので、
  %%$$$$$$$$$$$$$$$$$$$$$$$$$$$$$$$$$$$$$$$$$$$$$$$$$$$$$$$$$$$$$$$$$$$$$$
  \begin{eqnarray}
    m_h & = & m_a(1 - r^2)^2
  \\
    m & = & (m + m) (1-r^2)^2
  \\
    \frac{1}{2} & = & (1-r^2)^2
  \\
    \frac{1}{\sqrt{2}} & = & 1-r^2
  \\
    r^2 & = & 1 - \frac{1}{\sqrt{2}}
  \\
    r & = & \sqrt{1 - \frac{1}{\sqrt{2}}} = 0.5411961001461969...
  \end{eqnarray}
  %%$$$$$$$$$$$$$$$$$$$$$$$$$$$$$$$$$$$$$$$$$$$$$$$$$$$$$$$$$$$$$$$$$$$$$$
つまり、$v$ は光速の 0.54 倍程度である。


%%------------------------------
\subsubsection{参考:特殊相対性理論とニュートン力学の運動エネルギーの関係}
\label{sssec:宇宙船の加速:参考}
%% - - - - - - - - - - - - - - -

特殊相対性理論では、静止質量 $m_0$ である物体が速度 $v$ で移動している時、
その質量は以下のように変化する。
  %%$$$$$$$$$$$$$$$$$$$$$$$$$$$$$$$$$$$$$$$$$$$$$$$$$$$$$$$$$$$$$$$$$$$$$$
  \begin{eqnarray}
    m(v) & = & \frac{m_0}{\sqrt{1 - (v/c)^2}}
  \\
         & = & m_0 (1 - (v/c)^2)^{-1/2}
  \end{eqnarray}
  %%$$$$$$$$$$$$$$$$$$$$$$$$$$$$$$$$$$$$$$$$$$$$$$$$$$$$$$$$$$$$$$$$$$$$$$
これを $v=0$ でTaylor 展開すると:
  %%$$$$$$$$$$$$$$$$$$$$$$$$$$$$$$$$$$$$$$$$$$$$$$$$$$$$$$$$$$$$$$$$$$$$$$
  \begin{eqnarray}
    m(v) & = & m(0) + \frac{d m}{d v} v + \frac{1}{2} \frac{d^2 m}{d v^2} v^2
               + \cdots
  \\
    \frac{d m}{d v} & = &
       m_0 (-1/2) (1-(v/c)^2)^{-3/2} (-1) 2v/c^2
  \\                & = &
       (m_0/c^2)(1-(v/c)^2)^{-3/2} v
  \\
    \frac{d^2 m}{d v^2} & = &
       (m_0/c^2)(-3/2)(1-(v/c)^2)^{-5/2} (-2v/c^2) v
       + (m_0/c^2)(1-(v/c)^2)^{-3/2}
  \end{eqnarray}
  %%$$$$$$$$$$$$$$$$$$$$$$$$$$$$$$$$$$$$$$$$$$$$$$$$$$$$$$$$$$$$$$$$$$$$$$
$v=0$ の時の各微係数は以下の通り。
  %%$$$$$$$$$$$$$$$$$$$$$$$$$$$$$$$$$$$$$$$$$$$$$$$$$$$$$$$$$$$$$$$$$$$$$$
  \begin{eqnarray}
    \left.\frac{d m}{d v}\right|_{v=0} & = &
       0
  \\
    \left.\frac{d^2 m}{d v^2}\right|_{v=0} & = &
       \frac{m_0}{c^2}
  \end{eqnarray}
  %%$$$$$$$$$$$$$$$$$$$$$$$$$$$$$$$$$$$$$$$$$$$$$$$$$$$$$$$$$$$$$$$$$$$$$$
よって、速度$v$の時の物体の質量の増分 $\Delta m$は以下になる。
  %%$$$$$$$$$$$$$$$$$$$$$$$$$$$$$$$$$$$$$$$$$$$$$$$$$$$$$$$$$$$$$$$$$$$$$$
  \begin{eqnarray}
    \Delta m & = & m(v) - m(0)
  \\
      & \sim &
        \frac{d m}{d v} v
        +
        \frac{1}{2} \frac{d^2 m}{d v^2} v^2
  \\
      & = &
        \frac{m_0 v^2}{2 c^2}
  \end{eqnarray}
  %%$$$$$$$$$$$$$$$$$$$$$$$$$$$$$$$$$$$$$$$$$$$$$$$$$$$$$$$$$$$$$$$$$$$$$$
この増分質量をエネルギー変換すると以下になる。
  %%$$$$$$$$$$$$$$$$$$$$$$$$$$$$$$$$$$$$$$$$$$$$$$$$$$$$$$$$$$$$$$$$$$$$$$
  \begin{eqnarray}
    E_v & = &
      \Delta m c^2
  \\
      & \sim &
        \frac{m_0 v^2}{2}
  \end{eqnarray}
  %%$$$$$$$$$$$$$$$$$$$$$$$$$$$$$$$$$$$$$$$$$$$$$$$$$$$$$$$$$$$$$$$$$$$$$$
つまり、ニュートン力学における運動エネルギーとなる。

ただし、この関係が成立するのは、$v$ が十分小さいところであり、
光速 $c$ に近づくと Taylor 展開の3次以降の値が大きくなり、
質量増分すなわちエネルギー増分が大きくなり、
加速が難しくなる。

