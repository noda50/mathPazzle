%%----------------------------------------------------------------------
\section{三体に関する問題}
\label{s:三体に関する問題}
%% - - - - - - - - - - - - - - - - - - - - - - - - - - - - - - - - - - -

%%--------------------------------------------------
\subsection{宇宙船の加速}
\label{ssec:宇宙船の加速}
%% - - - - - - - - - - - - - - - - - - - - - - - - -
%%-----------------------------
\subsubsection{問題}
\label{sssec:宇宙船の加速:問題}
%% - - - - - - - - - - - - - -

地球から何光年も離れたある星までの宇宙旅行を考える。
(目的の星は、地球に対し相対速度 0 であるとしておく。)

質量 $m$ の宇宙船を加速し、速度 $v$ まで加速してから巡航し、
その後、減速して目的の星に着陸する。
星での調査の後、地球に向けて同様に加速・減速して帰還する。
この加速・減速に対し、超未来の技術により、
ほぼエネルギーロスゼロで加速・減速できるものとする。

このプランで、出発時に宇宙船と同じ質量 $m$ のエネルギー源を積んでいったとして、
光速 $c$ との速度比 $r = v/c$ の最大値を求めよ。

\paragraph*{ヒント}
相対性理論により、光速を $c$ とすると、
静止質量 $m_0$ の物体が速度 $v$ で動いている場合の質量 $m_v$ は以下の式となる。
  %%$$$$$$$$$$$$$$$$$$$$$$$$$$$$$$$$$$$$$$$$$$$$$$$$$$$$$$$$$$$$$$$$$$$$$$
  \begin{eqnarray}
    m_v & = & \frac{m_0}{\sqrt{1-\frac{v^2}{c^2}}}
  \end{eqnarray}
  %%$$$$$$$$$$$$$$$$$$$$$$$$$$$$$$$$$$$$$$$$$$$$$$$$$$$$$$$$$$$$$$$$$$$$$$

\clearpage
%%------------------------------
\subsubsection{解答}
\label{sssec:宇宙船の加速:解答}
%% - - - - - - - - - - - - - - -
(エネルギー保存則だけを考慮した場合)

エネルギー用物質の質量を $m_E$ としておく。

出発時の総重量 $m_a$ は、
  %%$$$$$$$$$$$$$$$$$$$$$$$$$$$$$$$$$$$$$$$$$$$$$$$$$$$$$$$$$$$$$$$$$$$$$$
  \begin{eqnarray}
    m_a & = & m + m_E
  \end{eqnarray}
  %%$$$$$$$$$$$$$$$$$$$$$$$$$$$$$$$$$$$$$$$$$$$$$$$$$$$$$$$$$$$$$$$$$$$$$$
往路の加速後の速度 $v$ における静止質量を $m_b$ としておく。
(加速によりエネルギーを使い、質量が減る。)
また、その速度での、地球運動系での質量を $m_c$ とする。
この時、エネルギー変換が理想的であるとすると、以下の式が成り立つ。
  %%$$$$$$$$$$$$$$$$$$$$$$$$$$$$$$$$$$$$$$$$$$$$$$$$$$$$$$$$$$$$$$$$$$$$$$
  \begin{eqnarray}
    m_c & = & m_a
  \label{eq:m-c-m-a}
  \\
    m_b & = & m_c \sqrt{1 - \frac{v^2}{c^2}} = m_a \sqrt{1 - r^2}
  \label{eq:m-b-m-c-v}
  \end{eqnarray}
  %%$$$$$$$$$$$$$$$$$$$$$$$$$$$$$$$$$$$$$$$$$$$$$$$$$$$$$$$$$$$$$$$$$$$$$$
目的の星に静止した時の静止質量を $m_d$ とし、
速度 $v$ での運動系における質量を $m_e$ とすると、以下の式となる。
  %%$$$$$$$$$$$$$$$$$$$$$$$$$$$$$$$$$$$$$$$$$$$$$$$$$$$$$$$$$$$$$$$$$$$$$$
  \begin{eqnarray}
    m_e & = & m_b
  \\
    m_d & = & m_e \sqrt{1 - \frac{v^2}{c^2}} = m_b \sqrt{1 - r^2}
  \\
        & = & m_a \sqrt{1 - r^2} \sqrt{1 - r^2} = m_a (1 - r^2)
  \label{eq:m-d-m-a-r}
  \end{eqnarray}
  %%$$$$$$$$$$$$$$$$$$$$$$$$$$$$$$$$$$$$$$$$$$$$$$$$$$$$$$$$$$$$$$$$$$$$$$
これを、復路でも繰り返すので、
復路の速度 $v$ での移動時の静止質量を $v_f$、
地球運動系での質量を $v_g$、
地球に静止した時の静止質量を $v_h$、
速度 $v$ の運動系から見た質量を $v_i$ とすると、
  %%$$$$$$$$$$$$$$$$$$$$$$$$$$$$$$$$$$$$$$$$$$$$$$$$$$$$$$$$$$$$$$$$$$$$$$
  \begin{eqnarray}
    m_g & = & m_d
  \\
    m_f & = & m_g \sqrt{1 - \frac{v^2}{c^2}} = m_d \sqrt{1 - r^2}
  \\
        & = & m_a (1 - r^2) \sqrt{1 - r^2}
  \label{eq:m-f-m-a-r}
  \\
    m_i & = & m_f
  \\
    m_h & = & m_i \sqrt{1 - \frac{v^2}{c^2}} = m_f \sqrt{1 - r^2}
  \\
        & = & m_a (1 - r^2) \sqrt{1 - r^2} \sqrt{1 - r^2} = m_a(1 - r^2)^2
  \label{eq:m-h-m-a-r}
  \end{eqnarray}
  %%$$$$$$$$$$$$$$$$$$$$$$$$$$$$$$$$$$$$$$$$$$$$$$$$$$$$$$$$$$$$$$$$$$$$$$
地球帰還時に、エネルギー用物質を使い果たしているとすると、$m_h = m$ である。
また、仮定により、$m_E = m$ であるので、
  %%$$$$$$$$$$$$$$$$$$$$$$$$$$$$$$$$$$$$$$$$$$$$$$$$$$$$$$$$$$$$$$$$$$$$$$
  \begin{eqnarray}
    m_h & = & m_a(1 - r^2)^2
  \\
    m & = & (m + m) (1-r^2)^2
  \\
    \frac{1}{2} & = & (1-r^2)^2
  \\
    \frac{1}{\sqrt{2}} & = & 1-r^2
  \\
    r^2 & = & 1 - \frac{1}{\sqrt{2}}
  \\
    r & = & \sqrt{1 - \frac{1}{\sqrt{2}}} = 0.5411961001461969...
  \end{eqnarray}
  %%$$$$$$$$$$$$$$$$$$$$$$$$$$$$$$$$$$$$$$$$$$$$$$$$$$$$$$$$$$$$$$$$$$$$$$
つまり、$v$ は光速の 0.54 倍程度である。

%%------------------------------
\subsubsection{解答2}
\label{sssec:宇宙船の加速:解答2}
%% - - - - - - - - - - - - - - -
(エネルギー保存則と運動量保存則を考慮した場合)

質量について、\secref{sssec:宇宙船の加速:解答}の変数をそのまま用いるとする。

出発時、宇宙船及び燃料は静止しているとする。
よって、系全体の運動量は 0 である。

往路において、速度 0 から速度 $\V{v}$ になった時、
その宇宙船の運動量 $\V{P}_c$ は以下となる。
  %%$$$$$$$$$$$$$$$$$$$$$$$$$$$$$$$$$$$$$$$$$$$$$$$$$$$$$$$$$$$$$$$$$$$$$$
  \begin{eqnarray}
    \V{P}_c & = & m_c \V{v}
  \label{eq:P-c-m-c-v}
  \end{eqnarray}
  %%$$$$$$$$$$$$$$$$$$$$$$$$$$$$$$$$$$$$$$$$$$$$$$$$$$$$$$$$$$$$$$$$$$$$$$
この運動量を打ち消すために、
速度 $\V{v}$ と逆方向 $\V{u}$ に進む光を仮定する。
その光のエネルギーを $E_{lc}$ とすると、その運動量 $\V{P}_{lc}$ は以下となる。
  %%$$$$$$$$$$$$$$$$$$$$$$$$$$$$$$$$$$$$$$$$$$$$$$$$$$$$$$$$$$$$$$$$$$$$$$
  \begin{eqnarray}
    \V{P}_{lc} & = & \frac{E_{lc} \V{u}}{c}
  \\
    \V{u} & = & - \frac{\V{v}}{\Abs{\V{v}}}
  \end{eqnarray}
  %%$$$$$$$$$$$$$$$$$$$$$$$$$$$$$$$$$$$$$$$$$$$$$$$$$$$$$$$$$$$$$$$$$$$$$$
ここで、エネルギー$E_{lc}$ を質量 $m_{lc}$ として換算すると、
  %%$$$$$$$$$$$$$$$$$$$$$$$$$$$$$$$$$$$$$$$$$$$$$$$$$$$$$$$$$$$$$$$$$$$$$$
  \begin{eqnarray}
    m_{lc} & = & \frac{E_{lc}}{c^2}
  \end{eqnarray}
  %%$$$$$$$$$$$$$$$$$$$$$$$$$$$$$$$$$$$$$$$$$$$$$$$$$$$$$$$$$$$$$$$$$$$$$$

最初の運動量 0 を保存するためには、
$\V{P}_c$ と $\V{P}_{lc}$ はバランスしてないと行けない。
よって、
  %%$$$$$$$$$$$$$$$$$$$$$$$$$$$$$$$$$$$$$$$$$$$$$$$$$$$$$$$$$$$$$$$$$$$$$$
  \begin{eqnarray}
    \V{P}_c + \V{P}_{lc}
      & = &
        0
  \\
    m_c v
      & = &
        \frac{E_{lc}}{c}
  \\
      & = &
        \frac{m_{lc} c^2}{c}
  \\
      & = &
        m_{lc} c
  \\
    m_{lc} & = & m_c \frac{v}{c}
  \end{eqnarray}
  %%$$$$$$$$$$$$$$$$$$$$$$$$$$$$$$$$$$$$$$$$$$$$$$$$$$$$$$$$$$$$$$$$$$$$$$
  

これらの質量分を \eqref{eq:m-c-m-a}、\eqref{eq:m-b-m-c-v}を組み込むと
以下になる。
  %%$$$$$$$$$$$$$$$$$$$$$$$$$$$$$$$$$$$$$$$$$$$$$$$$$$$$$$$$$$$$$$$$$$$$$$
  \begin{eqnarray}
    m_a
      & = &
        m_c + m_{lc}
  \\
      & = &
        \left(1 + \frac{v}{c}\right) m_c
  \\
      & = &
        (1 + r) m_c
  \\
    m_b & = & m_c \sqrt{1 - \frac{v^2}{c^2}}
  \\
        & = & 
           \frac{m_a \sqrt{1 - r^2}}{1+r} 
  \end{eqnarray}
  %%$$$$$$$$$$$$$$$$$$$$$$$$$$$$$$$$$$$$$$$$$$$$$$$$$$$$$$$$$$$$$$$$$$$$$$
同様の式変形が\eqref{eq:m-d-m-a-r}、\eqref{eq:m-f-m-a-r}、\eqref{eq:m-h-m-a-r}
でも適用されるので、以下のようになる。
  %%$$$$$$$$$$$$$$$$$$$$$$$$$$$$$$$$$$$$$$$$$$$$$$$$$$$$$$$$$$$$$$$$$$$$$$
  \begin{eqnarray}
    m_d
      & = & 
        \frac{m_a (1 - r^2)}{(1+r)^2}
  \\
    m_f
      & = & 
        \frac{m_a (1 - r^2)^{3/2}}{(1+r)^3}
  \\
    m_h
      & = & 
        \frac{m_a (1 - r^2)^2}{(1+r)^4}
  \label{eq:P:m-h-m-a-r}
  \end{eqnarray}
  %%$$$$$$$$$$$$$$$$$$$$$$$$$$$$$$$$$$$$$$$$$$$$$$$$$$$$$$$$$$$$$$$$$$$$$$
ここで、
初期のエネルギー源燃料の質量が宇宙船質量と同じ、
つまり $m_E=m$ とし、
帰還時、その燃料を使い果たしているとすると、
  %%$$$$$$$$$$$$$$$$$$$$$$$$$$$$$$$$$$$$$$$$$$$$$$$$$$$$$$$$$$$$$$$$$$$$$$
  \begin{eqnarray}
    m_a
      & = & 
        m + m_E = 2m
  \\
    m_h
      & = & 
        m
  \\
    m_h
      & = &
        (1/2) m_a
  \end{eqnarray}
  %%$$$$$$$$$$$$$$$$$$$$$$$$$$$$$$$$$$$$$$$$$$$$$$$$$$$$$$$$$$$$$$$$$$$$$$
これを\eqref{eq:P:m-h-m-a-r}に代入して $r$ を求めると以下になる。
  %%$$$$$$$$$$$$$$$$$$$$$$$$$$$$$$$$$$$$$$$$$$$$$$$$$$$$$$$$$$$$$$$$$$$$$$
  \begin{eqnarray}
    \frac{1}{2}
      & = &
        \frac{(1 - r^2)^2}{(1+r)^4}
  \\
    (1+r)^4
      & = &
        2 (1-r^2)^2
  \\
    r^4 - 4r^3 - 10r^2 - 4r + 1
      & = & 0
  \\
    (r+1)^2 (r^2 - 6 + 1)
      & = & 0
  \\
    r
      & = & -1 \mbox{ or } 3 \pm 2\sqrt{2}
  \end{eqnarray}
  %%$$$$$$$$$$$$$$$$$$$$$$$$$$$$$$$$$$$$$$$$$$$$$$$$$$$$$$$$$$$$$$$$$$$$$$
$0 < r < 1$ であるので、
  %%$$$$$$$$$$$$$$$$$$$$$$$$$$$$$$$$$$$$$$$$$$$$$$$$$$$$$$$$$$$$$$$$$$$$$$
  \begin{eqnarray}
    r
      & = & 3 - 2 \sqrt{2}
  \\
      & \sim & 0.1715728752538097 \ldots
  \end{eqnarray}
  %%$$$$$$$$$$$$$$$$$$$$$$$$$$$$$$$$$$$$$$$$$$$$$$$$$$$$$$$$$$$$$$$$$$$$$$

また、燃料質量 $m_E$ を宇宙船質量の $k$ 倍としておくと、以下のようになる。
  %%$$$$$$$$$$$$$$$$$$$$$$$$$$$$$$$$$$$$$$$$$$$$$$$$$$$$$$$$$$$$$$$$$$$$$$
  \begin{eqnarray}
    \frac{1}{1+k}
      & = &
        \frac{(1 - r^2)^2}{(1+r)^4}
  \\
    1+k
      & = &
        \frac{(1+r)^4}{(1-r^2)^2}
  \\
      & = &
        \frac{(1+r)^4}{(1+r)^2(1-r)^2}
  \\
      & = &
        \frac{(1+r)^2}{(1-r)^2}
  \\
    k
      & = &
        \frac{(1+r)^2}{(1-r)^2} - 1
  \\
      & = &
        \frac{(1+r)^2 - (1-r)^2}{(1-r)^2}
  \\
      & = &
        \frac{4r}{(1-r)^2}
  \end{eqnarray}
  %%$$$$$$$$$$$$$$$$$$$$$$$$$$$$$$$$$$$$$$$$$$$$$$$$$$$$$$$$$$$$$$$$$$$$$$
つまり、光速の半分まで加速しながら帰ってくる宇宙旅行をするためには
宇宙船質量の 8 倍の燃料を積む必要がある。
光速の 90\% の場合は、360倍の燃料となる。

  
%%------------------------------
\subsubsection{参考:特殊相対性理論とニュートン力学の運動エネルギーの関係}
\label{sssec:宇宙船の加速:参考}
%% - - - - - - - - - - - - - - -

特殊相対性理論では、静止質量 $m_0$ である物体が速度 $v$ で移動している時、
その質量は以下のように変化する。
  %%$$$$$$$$$$$$$$$$$$$$$$$$$$$$$$$$$$$$$$$$$$$$$$$$$$$$$$$$$$$$$$$$$$$$$$
  \begin{eqnarray}
    m(v) & = & \frac{m_0}{\sqrt{1 - (v/c)^2}}
  \\
         & = & m_0 (1 - (v/c)^2)^{-1/2}
  \end{eqnarray}
  %%$$$$$$$$$$$$$$$$$$$$$$$$$$$$$$$$$$$$$$$$$$$$$$$$$$$$$$$$$$$$$$$$$$$$$$
これを $v=0$ でTaylor 展開すると:
  %%$$$$$$$$$$$$$$$$$$$$$$$$$$$$$$$$$$$$$$$$$$$$$$$$$$$$$$$$$$$$$$$$$$$$$$
  \begin{eqnarray}
    m(v) & = & m(0) + \frac{d m}{d v} v + \frac{1}{2} \frac{d^2 m}{d v^2} v^2
               + \cdots
  \\
    \frac{d m}{d v} & = &
       m_0 (-1/2) (1-(v/c)^2)^{-3/2} (-1) 2v/c^2
  \\                & = &
       (m_0/c^2)(1-(v/c)^2)^{-3/2} v
  \\
    \frac{d^2 m}{d v^2} & = &
       (m_0/c^2)(-3/2)(1-(v/c)^2)^{-5/2} (-2v/c^2) v
       + (m_0/c^2)(1-(v/c)^2)^{-3/2}
  \end{eqnarray}
  %%$$$$$$$$$$$$$$$$$$$$$$$$$$$$$$$$$$$$$$$$$$$$$$$$$$$$$$$$$$$$$$$$$$$$$$
$v=0$ の時の各微係数は以下の通り。
  %%$$$$$$$$$$$$$$$$$$$$$$$$$$$$$$$$$$$$$$$$$$$$$$$$$$$$$$$$$$$$$$$$$$$$$$
  \begin{eqnarray}
    \left.\frac{d m}{d v}\right|_{v=0} & = &
       0
  \\
    \left.\frac{d^2 m}{d v^2}\right|_{v=0} & = &
       \frac{m_0}{c^2}
  \end{eqnarray}
  %%$$$$$$$$$$$$$$$$$$$$$$$$$$$$$$$$$$$$$$$$$$$$$$$$$$$$$$$$$$$$$$$$$$$$$$
よって、速度$v$の時の物体の質量の増分 $\Delta m$は以下になる。
  %%$$$$$$$$$$$$$$$$$$$$$$$$$$$$$$$$$$$$$$$$$$$$$$$$$$$$$$$$$$$$$$$$$$$$$$
  \begin{eqnarray}
    \Delta m & = & m(v) - m(0)
  \\
      & \sim &
        \frac{d m}{d v} v
        +
        \frac{1}{2} \frac{d^2 m}{d v^2} v^2
  \\
      & = &
        \frac{m_0 v^2}{2 c^2}
  \end{eqnarray}
  %%$$$$$$$$$$$$$$$$$$$$$$$$$$$$$$$$$$$$$$$$$$$$$$$$$$$$$$$$$$$$$$$$$$$$$$
この増分質量をエネルギー変換すると以下になる。
  %%$$$$$$$$$$$$$$$$$$$$$$$$$$$$$$$$$$$$$$$$$$$$$$$$$$$$$$$$$$$$$$$$$$$$$$
  \begin{eqnarray}
    E_v & = &
      \Delta m c^2
  \\
      & \sim &
        \frac{m_0 v^2}{2}
  \end{eqnarray}
  %%$$$$$$$$$$$$$$$$$$$$$$$$$$$$$$$$$$$$$$$$$$$$$$$$$$$$$$$$$$$$$$$$$$$$$$
つまり、ニュートン力学における運動エネルギーとなる。

ただし、この関係が成立するのは、$v$ が十分小さいところであり、
光速 $c$ に近づくと Taylor 展開の3次以降の値が大きくなり、
質量増分すなわちエネルギー増分が大きくなり、
加速が難しくなる。

%%------------------------------
\subsubsection{参考:特殊相対性理論とニュートン力学の運動量の関係}
\label{sssec:宇宙船の加速:参考2}
%% - - - - - - - - - - - - - - -

特殊相対性理論に於いても、
速度 $\V{v}$ で動いている質量 $m$ の物体の運動量 $P_v$ は以下で表される。
  %%$$$$$$$$$$$$$$$$$$$$$$$$$$$$$$$$$$$$$$$$$$$$$$$$$$$$$$$$$$$$$$$$$$$$$$
  \begin{eqnarray}
    \V{P}_v & = & m \V{v}
  \end{eqnarray}
  %%$$$$$$$$$$$$$$$$$$$$$$$$$$$$$$$$$$$$$$$$$$$$$$$$$$$$$$$$$$$$$$$$$$$$$$
ただし、質量 $m$ は静止質量 $m_0$ ではなく運動質量である。
よって、静止質量で書いた運動量は以下の通り。
  %%$$$$$$$$$$$$$$$$$$$$$$$$$$$$$$$$$$$$$$$$$$$$$$$$$$$$$$$$$$$$$$$$$$$$$$
  \begin{eqnarray}
    \V{P}_v & = & m_0 \frac{\V{v}}{\sqrt{1-(v/c)^2}}
  \end{eqnarray}
  %%$$$$$$$$$$$$$$$$$$$$$$$$$$$$$$$$$$$$$$$$$$$$$$$$$$$$$$$$$$$$$$$$$$$$$$

一方、エネルギー $E_l$ で $\V{u}$ の方向に進む
光の運動量 $\V{P}_l$ は以下で表される。
  %%$$$$$$$$$$$$$$$$$$$$$$$$$$$$$$$$$$$$$$$$$$$$$$$$$$$$$$$$$$$$$$$$$$$$$$
  \begin{eqnarray}
    \V{P}_l & = & \frac{E\V{u}}{c}
  \\
    \mbox{, where} & & \Abs{\V{u}} = 1
  \end{eqnarray}
  %%$$$$$$$$$$$$$$$$$$$$$$$$$$$$$$$$$$$$$$$$$$$$$$$$$$$$$$$$$$$$$$$$$$$$$$
  
